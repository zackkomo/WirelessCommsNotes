\documentclass[12pt]{report} % add 'draft' to suppress figures
% Double-spaced, single-column format for IEEE journal submission
% \documentclass[12pt,journal,draftclsnofoot,onecolumn]{IEEEtran} 
\usepackage{wrapfig,booktabs,fancyhdr,amsmath,amsfonts,tabularx,numprint}
\usepackage{cite,bm,bbm,amssymb,amsthm,url,multirow,times,enumitem,comment}
\usepackage{mathtools,siunitx,balance,tikz,adjustbox,graphicx,array}
\usepackage[font=footnotesize]{subcaption}
\usepackage[linesnumbered,ruled]{algorithm2e}
\usepackage[colorlinks=true, linkcolor=black, citecolor=blue, urlcolor=blue]{hyperref}
\usepackage{geometry}
 \geometry{
 a4paper,
 total={170mm,257mm},
 left=20mm,
 top=20mm,
 }
 \usepackage[acronym,nomain,nonumberlist]{glossaries}
 \makeglossaries

 \newcommand{\vb}{\boldsymbol}
\newcommand{\vbh}[1]{\hat{\boldsymbol{#1}}}
\newcommand{\vbc}[1]{\check{\boldsymbol{#1}}}
\newcommand{\vbb}[1]{\bar{\boldsymbol{#1}}}
\newcommand{\vbt}[1]{\tilde{\boldsymbol{#1}}}
\newcommand{\vbs}[1]{{\boldsymbol{#1}}^*}
\newcommand{\vbd}[1]{\dot{{\boldsymbol{#1}}}}
\newcommand{\abs}[1]{\left|{#1}\right|}
\newcommand{\by}{\times}
\newcommand{\tr}{\mathsf{T}}
\newcommand{\sfrac}[2]{\textstyle \frac{#1}{#2}}
\newcommand{\ba}{\begin{array}}
\newcommand{\ea}{\end{array}}
\newcommand{\sinc}{\text{sinc}}
\newcommand{\define}{\triangleq}
%\newcommand{\define}{\coloneqq}
\newcommand{\cnr}{C/N_0}
\newcommand{\sgn}{\text{sgn}}
\renewcommand{\Re}{\mathbb{R}}
\renewcommand{\Im}{\mathbb{I}}
\newcommand{\E}[1]{\mathbb{E}\left[ #1 \right]}
\newcommand{\brm}[1]{\bm{\text{#1}}}
\newcommand{\round}[1]{\ensuremath{\lfloor#1\rceil}}
\DeclareMathAlphabet{\mathpzc}{OT1}{pzc}{m}{it}
\DeclareMathOperator*{\argmin}{\arg\!\min}
\DeclareMathOperator*{\argmax}{\arg\!\max}

% Macros for this paper
\newcommand{\Fc}[1]{\ensuremath{F_{\text{c}#1}}}
\newcommand{\Fs}{\ensuremath{F_\text{s}}}
\newcommand{\Fsr}{\ensuremath{F_\text{sr}}}

\newacronym{PL}{PL}{Path loss}
\newacronym{LOS}{LOS}{Line of sight}
\newacronym{NLOS}{NLOS}{Non-line of sight}



% \topmargin = 0 mm 
% \oddsidemargin = -1 mm 
% \evensidemargin = -1 mm
% \headheight = 0 mm 
% \headsep = 8 mm 
% \textheight = 220 mm 
% \textwidth =170 mm 
% \parindent = 0 mm
% \parskip = 4 mm 

\begin{document}
%%% ----------------------------------------------------------Front Matter
\title{An Exegesis of Ontological Hermeneutics}

\tableofcontents
\chapter{Path Loss Models}

\gls{PL} models are formulas which allow for modeling the received
power $P_r$ given the transmit power $P_t$, distance $d$, and signal carrier
wavelength $\lambda$. The resulting \gls{PL} is due to signal propagation
through the environment.

\section{Friss Formula}
For a transmitter and receiver spaced $d$ distance apart, the received power
$Pr$ follows:

\begin{align}
  \label{eq:general_PrPt}
  P_r = P_t \frac{1}{4\pi d} A_e
\end{align}

where $A_e = \frac{\lambda}{4 \pi}$ for an omnidirectional antenna and a signal
with a wavelength $\lambda$. Substituting $A_e$ in (\ref{eq:general_PrPt})
provides the Friis Formula, which applies for a wireless channel in a vacuum.
Further adding terms $G_t$ and $G_r$ for the transmit and receive antenna gains,
we reach the generalized Friis formula.

\begin{align}
  \label{eq:friss_formula}
  P_r = \frac{G_t G_r \lambda ^2 P_t}{(4\pi d)^2}
\end{align}

\section{Path Loss Exponent Model}
Since wireless channels exist in environments without a vacuum, a generalized
model is needed to cover a variety of cases. A generalized such model is an
exponential model using the received power formula below.

\begin{align}
  \label{eq:general_exp_PL}
  P_r = P_t G_t G_r K \left[\frac{d_0}{d}\right]^\alpha
\end{align}

The parameter $d_0$ is a unit normalization parameter, usually set to $1$ m. The
parameter $\alpha$ is the \gls{PL} exponent. The parameter $K$ is the nominal
\gls{PL} at $1$ m. Notice when $K = [\frac{\lambda}{4 \pi}]^2$ defined as $K_0$
and $\alpha = 2$, the formula reveals the Friis Formula from
(\ref{eq:friss_formula}).

\chapter{Multipath Effect Modeling}
Other than the natural landscape of the environment, there are other effects
that may further increase \gls{PL}.

\section{Blocking}
Blocking allows for a multi-slope \gls{PL} model. Essentially, the \gls{PL}
model has multiple definitions, and follows each one with a certain probability.
This usually translates to following a certain \gls{PL} for a \gls{LOS} path,
and following another model for the \gls{NLOS} path. A common model for this is
the Exponential Blocking model

\begin{align}
  \label{eq:exp_blocking}
  P_r = \left\{ \begin{array}{ll}
                  PL_{LOS}  & \text{w.p.} ~~~~~~ e^{-d/\beta}  \\
                  PL_{NLOS} & \text{w.p.} ~~~ 1 - e^{-d/\beta} \\
                \end{array}
  \right.
\end{align}

where $\beta$ acts as the mean distance before blocking occurs (i.e. by a building), and the $PL_{LOS}$ and $PL_{NLOS}$ terns are functions for \gls{PL} models.

\subsection*{Blocking example}
Imagine a scenario in which a \gls{LOS} path is in free space and the \gls{NLOS} is modeled as an exponential \gls{PL} with an exponent $\alpha = 2.5$ and a distance of $d = 100$ between the transmitter and receiver. The blocking exponent $\beta = 25$. What does the $SNR = P_r/P_t$ model finalize to if $G = G_t G_r = 1$, $K_{\text{LOS}} = K_{\text{NLOS}} = K_0 = -40$ dB and $d_0 = d_1 = 1$.
\\
\\
Answer:

\begin{align}
  SNR = \frac{P_r}{P_t} = \left\{ \begin{array}{ll}
                                    K_0 \left[\frac{1}{100}\right]^2     & \text{w.p.} ~~~~~~ e^{-100/25}  \\
                                    K_0 \left[\frac{1}{100}\right]^{2.5} & \text{w.p.} ~~~ 1 - e^{-100/25} \\
                                  \end{array}
  \right. \nonumber
  = \left\{ \begin{array}{ll}
              -80 \text{ dB} & \text{w.p.} ~ 0.0183 \\
              -90 \text{ dB} & \text{w.p.} ~ 0.9817 \\
            \end{array}
  \right. \nonumber
\end{align}

\section{Shadowing}
Shadowing is attenuation caused by objects between the transmitter and
receiver that reduce the received signal's power. It is modeled as random and does not depend on the distance of the wireless communication link. By modeling shadowing we account for the constructive and destructive interference cause by the transmitted signal reflecting, diffracting and scattering, the components of which are called multipath components.

The shadowing effect is modeled as lognormal. The received power equation from a \gls{PL} model is simply multiplied with a new variable $\mathcal{X}$, where $x_{\text{dB}} = 10 log_{10}{\mathcal{X}} \sim \mathcal{N}(0,\sigma_{\text{dB}}^2)$.

Alternatively, some use a model where $P_r = P_t \Psi$ where $ \psi_{\text{dB}} = 10 log_{10}{\Psi}  \sim \mathcal{N}(PL_{\text{dB}}(d),\sigma_{\text{dB}}^2)$ for some \gls{PL} model.

\subsection*{Shadowing example}
Imagine a scenario in which we wish to evaluate Wi-Fi coverage for a desired range of $100$ m, with a minimum $SNR = 5$ dB and outage constraint at $1\%$ for the $5$ GHz frequencies. The \gls{PL} is modeled as exponential with $\alpha = 3$, $K = K_0 = -50$ dB (for $F_c \approx 5$ GHz), no TX and RX gain, a noise power of $P_n = -100$ dBm and a shadowing standard deviation of $\sigma_{\text{dB}} = 6$. What is the mimum transmition power to meet the outage constraint?
\\
\\
Answer:

\begin{align}
  SNR             & = \frac{P_r}{P_t} \geq 5 \text{ dB} \nonumber                                       \\
  P_r             & \geq -95 \text{ dB} \nonumber                                                       \\
  P_r             & = P_{t,\text{dB}} + K_{0,\text{dB}} + x_{\text{dB}} - \alpha 10 log_{10}d \nonumber \\
  P_{t,\text{dB}} & \geq -95 + 50 - x_{\text{dB}} + 60 \nonumber                                        \\
  P_{t,\text{dB}} & \geq 15 - x_{\text{dB}} \nonumber
\end{align}

Since $x_{\text{dB}}$ is normally distributed, we simply use the CDF of a Gaussian distribution to evaluate transmit power that would allow for the required outage constraint of $0.01$:

\begin{align}
  P(P_{t,\text{dB}} \geq  15 - x_{\text{dB}}) = 1 - 0.01 \nonumber \\
  15 - P_{t,\text{dB}} = \sigma_{\text{dB}}Q^{-1}(0.99) \nonumber  \\
  P_{t,\text{dB}} \geq 29 \text{ dBm} \nonumber
\end{align}

\chapter{Multipath Channel Models}

\glsaddall
\printglossary[type=\acronymtype,title=Acronyms]

% \bibliographystyle{ieeetran}
% \bibliography{ref}
\end{document}

