\documentclass[12pt]{report}
\usepackage{color,soul}
\usepackage{subcaption,wrapfig,graphicx,booktabs,fancyhdr,amsmath,amsfonts}
\usepackage{cite,bm,amssymb,amsthm,url,multirow,times,enumitem,comment}
\usepackage{mathtools,siunitx,tikz}
\usepackage[linesnumbered,ruled]{algorithm2e}
\usepackage[colorlinks=true, linkcolor=black, citecolor=blue, urlcolor=blue]{hyperref}
\newcommand{\vb}{\boldsymbol}
\newcommand{\vbh}[1]{\hat{\boldsymbol{#1}}}
\newcommand{\vbc}[1]{\check{\boldsymbol{#1}}}
\newcommand{\vbb}[1]{\bar{\boldsymbol{#1}}}
\newcommand{\vbt}[1]{\tilde{\boldsymbol{#1}}}
\newcommand{\vbs}[1]{{\boldsymbol{#1}}^*}
\newcommand{\vbd}[1]{\dot{{\boldsymbol{#1}}}}
\newcommand{\abs}[1]{\left|{#1}\right|}
\newcommand{\by}{\times}
\newcommand{\tr}{\mathsf{T}}
\newcommand{\sfrac}[2]{\textstyle \frac{#1}{#2}}
\newcommand{\ba}{\begin{array}}
\newcommand{\ea}{\end{array}}
\newcommand{\sinc}{\text{sinc}}
\renewcommand{\equiv}{\triangleq}
\newcommand{\cnr}{C/N_0}
\newcommand{\sgn}{\text{sgn}}
\renewcommand{\Re}{\mathbb{R}}
\renewcommand{\Im}{\mathbb{I}}
\newcommand{\E}[1]{\mathbb{E}\left[ #1 \right]}
\DeclareMathAlphabet{\mathpzc}{OT1}{pzc}{m}{it}
\DeclareMathOperator*{\argmin}{\arg\!\min}
\DeclareMathOperator*{\argmax}{\arg\!\max}

\topmargin = 0 mm 
\oddsidemargin = -1 mm 
\evensidemargin = -1 mm
\headheight = 0 mm 
\headsep = 8 mm 
\textheight = 220 mm 
\textwidth =170 mm 
\parindent = 0 mm
\parskip = 4 mm 


\begin{document}

\section*{Reviewer 1}
\subsection*{General Comments:}

{\textit{This is a very nice, thorough piece of work that develops on the
    authors' previous contributions to GNSS spoofing detection. The paper
    expands on the previously introduced ``sandwich'' detector, which utilises
    both overall power level monitoring and per channel correlation distortion
    monitoring to classify each received signal as belonging to one of the 4
    classes: nominal, multipath, jammed, or spoofed.}}

{\textit{The development is clear and the text is well structured and well
    written and deserves to be published with some largely minor
    modifications.}}
  
{Thank you for your assessment and helpful comments.}
  
  
\subsection*{Suggested Improvements:}

{\textit{One area that could do with some clarification relates to
      Section IV A -
Received Power Level Measurement, in particular Eqn
      (5). In the preceding
development leading up to this section the
      authors have considered a single
signal only, which is both appropriate
      and simplifies the notation. However,
when considering the overall
      power in the receiver, the chosen notation and
description becomes
      somewhat convoluted. I think it would clarify the matter
here to
      introduce the total power as the sum of the power from all
      satellites
(and all signals within the L1 band from all satellites),
      the power from all
interference sources (jammers/spoofers etc) and the
      cross terms (between
satellites, between interferers and between
      satellites and interferers). In
this manner it is not necessary to
      decide whether an interferer should be
classified as part of the
      MAI. With this fuller notation, needed only for the
power level
      measurement, it would be clear where each contribution to the
overall
      measured power comes from.}}

  
  { This is a good suggestion.  We have adopted it fully in the revised
    Section IV.  The new Eq. (6) expresses the combined signal as the
    summation of all authentic and interference signals, plus noise.  $P_k$ is
    now defined in terms of this combined signal (cf. Eq. (7)), and the
    mean of $P_k$ (in linear units) is expressed as a sum of constituents,
    including cross-terms between authentic and interference signals
    (cf. Eq. (8)).  No cross-terms between interferers themselves is
    necessary in Eq. (8), as interferers are assumed to be mutually
    independent, as now explained in the paragraph just before (8).}
  

{\textit{In addition, I would argue that Eqn (6) is not strictly
      correct, as the right
hand side represents not the Power Spectral
      Density due to MAI, but rather the
total power due to MAI. Indeed the
      right hand side is the closed form solution
of the so-called spectral
      separation coefficient for BPSK(1) assuming infinite
transmit and
      receive bandwidths. Using Eqn (6) as is would greatly over
estimate the
      contribution of MAI to total received power.}}

  
  {We agree, and are grateful that you caught this error.  The revised
    expression for $P_k$ (Eqs. (7) and (8)) does not attempt approximation via
    $M_0$.  The earlier approach was equivalent to the current (correct)
    approach when $W_P = 3/(2\tau_c)$, but, as $W_P$ is assumed larger than
    $3/(2\tau_c)$, our former approach exaggerated the contribution of
    multi-access signals to $P_k$.  We have corrected our code and re-run the
    paper's simulations and tests against empirical data.}
  

{\textit{This leads to my final concern with this section: I don't really
      see how matched
power spoofing can lead to a total power delta of
      approximately 2 dB in a 2 MHz
bandwidth as shown in Fig. 4. By my
      calculation this would require an average
C/No of about 50 dB-Hz from
      approximately 20 satellites simultaneously.
I presume the large delta
      in Fig. 4 is largely due to the RF image energy
introduced by the
      spoofer. My concern then is that this same spoofer has been
used in the
      live data experiments also. Is this justifiable? Is it likely that
such
      a large amount of in-band noise would be generated by a "real world"
      spoofer?
}}
  
  We have revisited our empirical measurements and our calculations and can
  re-affirm the $\sim$2 dB uptick in the 2-MHz bandwidth at the beginning of
  spoofing in TEXBAT Scenario 3.  We were also surprised at the size of the
  increase.  It can be explained by noting that, for the high-quality antenna
  (Trimble Geodetic Zephyr II) and receiver (a 16-bit National Instruments
  front end) in question, the total authentic power in the 2-MHz bandwidth
  under nominal (clean) conditions was measured to be approximately -142 dBW.
  This is the combined power of $M_s + 1 = 12$ authentic GPS L1 C/A signals
  (plus a small amount of power from GPS L1(P)).

  This is a surprisingly large combined power.  Assuming a noise floor of -204
  dBW/Hz (thermal noise only, no multi-access noise), the total noise power in
  a 2-MHz bandwidth would be $63 - 204 = -141$ dBW.  Thus, the combined power
  of the authentic signals (and mild multipath) for this antenna/receiver
  combination is nearly equal to the total noise power in band.  One can
  verify this by examining the new Fig. 3, which shows the nominal (clean)
  spectrum in dark gray.  Selective integration of the spectrum reveals that
  the power in the center 2 MHz band is indeed twice that of the power in a 2
  MHz band of the noise floor (which one can approximate by centering a 2 MHz
  band at -4 MHz).

    The upshot of this is that, \emph{relative to the noise floor alone} (no
    multi-access interference), of the 12 authentic signals, there are two
    signals with $C/N_0 \approx 55$ dB-Hz (PRNs 13 and 23), two others with
    $C/N_0 \approx 53$ dB-Hz (PRNs 3 and 16), and three others above
    $C/N_0 \approx 50$ dB-Hz, with many of the remaining signals in the mid-
    to high-40 dB-Hz range.  Note that the \emph{measured} $C/N_0$ values are
    approximately 3 dB lower than this due to multi-access interference. (In a
    typical receiver, measured $C/N_0$ is actually $C/(N_0 + M_0)$.)

    Note also that, for TEXBAT Scenario 3, $\eta_i = 1.3$ for all $i$.  Thus,
    although Scenario 3 is advertised as a ``matched power'' scenario, it is
    only approximately so.  A truly matched scenario would have $\eta_i = 1$
    for all $i$.

    The reviewer is correct that some wideband quantization noise and aliasing
    due to the rudimentary RF front-end enters the 2 MHz band around L1 for
    TEXBAT Scenario 3.  But the power of such noise is small compared to the
    power in the spoofing signals of Scenario 3.  The quantization and
    aliasing noise does tend to make TEXBAT Scenarios 1-6 easier to detect
    than a more sophisticated spoofing attack would be (because they have
    slightly higher power).  But TEXBAT Scenario 7 does not suffer from such
    quantization and aliasing noise, nor do any of the Monte-Carlo-simulated
    scenarios against which the PD detector was tested.

    In any case, to avoid any confusion, and to avoid distracting the reader
    with excuses for the rudimentary front end used in TEXBAT Scenarios 1
    through 6, we have changed Figs. 3 and 4 to reflect the spoofing scenario
    from TEXBAT Scenario 7, which is free of aliasing or quantization noise.
    We have also re-drawn the nominal (clean) plots to make them clearer.

    Furthermore, to call attention to the extra noise in TEXBAT Scenarios 2
    through 6, we have modified the second paragraph of VII.A to read

    \begin{quotation}
      Note that {\tt tb2}--{\tt tb6} exhibit a modest amount of quantization
      and aliasing noise that makes them easier to detect than would be
      expected based on this paper's models, whereas {\tt tb7} is free from
      such extraneous noise.  {\tt tb7} is also special in that the spoofer
      exercises control of $\Delta \theta$, permitting nulling, as described
      in Section V-A, in the early stages of the attack.  In all other TEXBAT
      recordings, the spoofer controlled $\eta$ and $\Delta \tau$ but left
      $\Delta \theta$ at an arbitrary constant value ({\tt tb3}, {\tt tb4},
      and {\tt tb6}) or allowed it to ramp consistent with the pull-off rate
      ({\tt tb2} and {\tt tb5}).
    \end{quotation}
    
  
  
{\textit{Another area that could be made clearer is that the aim of
      this paper is not
so much to detect spoofing per se, but rather to
      detect a carry-off attack.
Thus H2 is the hypothesis that a carry-off
      attack is underway. This pre?supposes
that the receiver is initially
      tracking true GPS signals. This is a small point
but it is important to
      be clear on this.}}


 We agree that this clarification is warranted, and have modified the
  abstract, the introduction, and the conclusion to make clear that our
  spoofing detection applies to carry-off-type attacks.




\subsection*{Minor Comments:}

{\textit{Introduction: "... no civil GNSS signals yet incorporate cryptographic modulation...
" \\
      perhaps better to say "... no open civil GNSS signals ..." as the
      Galileo
Commercial Service is indeed a civil GNSS signal incorporating
      cryptographic
modulation.}}

  
  {We have incorporated this clarification in our revised manuscript.}
  


{\textit{P. 11: Under H2: Spoofing. Here you do not permit advance of the
      spoofed signal
with respect to the authentic signal, why?  Also, you
      state that spoofing where
the time difference exceeds two chips "is
      more properly classified as jamming".
I would argue that under certain
      circumstances this could well be classified
as very successful
      spoofing.}}

  
   Thank you for pointing this out.  We unwittingly left off the absolute
    value bars around $\Delta \tau$ here and in the description of $H_3$.  We
    have now fixed this, indicating that spoofing can be either advanced or
    delayed wrt the authentic signal.  We have also modified the text to
    justify our classification of spoofing as jamming when $|\Delta \tau|$ is
    sufficiently large:

    \begin{quotation}
      These bounds make sense because the spoofing signal must be at least as
      powerful as the authentic signal for reliable spoofing, and because
      spoofing whose $|\Delta \tau|$ value exceeds $\Delta \tau_1 = 2\tau_c$
      is no longer classified as carry-off-type spoofing, since the
      interference is uncorrelated with the authentic signal.  Instead, for
      $|\Delta \tau| \geq \Delta \tau_1$, the attack is classified as jamming.
    \end{quotation}

    We have left the assumed distribution of $\Delta \tau$ under $H_2$ as it
    was, since the worst case attack occurs when the spoofer mimics multipath.
  

{\textit{Figure 10 ? there's a floating exclamation mark (!) after
      the figure ? most
likely a TeX float placement command gone
      wrong...}}

  
  {{We have fixed this.  You were correct that it was a placement command
        gone wrong.}}
  



\section*{Reviewer 2}
\subsection*{Suggestions for Improvement:}

{\textit{The paper is generally well written and organized; however, the
      notation used is sometimes distracting. The proposed power-distortion
      detector is very interesting in that it can detect both high-power and
      low-power spoofing attempts. The authors acknowledge that the monitor is
      ineffective in cases where the spoofer is able to block or null the
      authentic GPS signals (which is difficult to accomplish). However, in
      the results section, while discussing Table 3, it is stated that the
      proposed detector is most effective during the initial spoofer
      pull-off. Beyond a certain point, the distortion will subside, and the
      monitor's effectiveness would decrease.}}

  
   Agreed.  We have modified the paper's abstract, introduction, and
    conclusion to emphasize the fact that its spoofer detection is designed to
    work for carry-off spoofing attacks; in particular, during the initial
    stages of a carry-off attack when distortion remains significant. 
  

{\textit{It would be interesting to see how very slow spoofer pull-off
      would affect the monitor, and what the resulting minimum detectable
      error would be. It appears unlikely that the proposed monitor can be
      completely fooled by a slow pull-off rate. However, the experimental
      results section does not mention details regarding the spoofer's
      pull-off rate. If it is very slow, is seems possible that the detector
      would initially classify the spoofing as multipath. Then, beyond a
      certain minimum detectable error, the detector would correctly identify
      the spoofing attempt. It is possible that the reason why results in
      Table 3 show no instances where spoofing is classified as multipath is
      because of a sufficiently quick pull-off rate.}}

  
   The reviewer is correct that close-in (small $|\Delta t|$),
    low-powered spoofing could fool the PD detector into declaring multipath
    ($H_1$).  However, a slow pull-off would afford the PD detector
    \emph{more} opportunities to observe distortion and declare the attack as
    spoofing than a fast pull-off attack.  We believe that the primary reason
    no empirical spoofing scenarios were mis-classified as multipath is that
    they were all designed to be \emph{effective} spoofing attacks, with
    sufficiently large $\eta$ to effect pull-off of the tracking points.
    Thus, while they were sometimes confused for jamming, their received power
    was too strong to be confused with multipath.

    The three references to TEXBAT data given in Section VII-A offer details
    on the pull-off rates of each attack.  In particular, {\tt tb7} has a
    particularly slow pull-off, whereas {\tt tb2} throuth {\tt tb6} are faster.
  
  


\subsection*{Minor Comments:}

{\textit{Variable `C' is used to represent the C/A code, and later used
      to represent the cost function}}
  
  {{We have modified $C(t)$ to read $C_r(t)$ throughout to avoid this confusion.}}
  

{\textit{letter 'j' is used to denote the imaginary unit on pg. 5,
      as well as a time index (pg. 7), and parameter space index
      (pg. 22) }}

  
  {{ Thanks for pointing this out. However, in this context, we do feel that
      the meaning of our notation is clear. That is, $j$ is often used as an
      indexing variable, and we have already used $i$ and $k$ elsewhere in the
      paper. Likewise, either $i$ or $j$ are also traditionally used to
      indicate imaginary components. Given that there is no overlap within the
      same expression, we'd prefer to keep the notation as is in this case.}}
  

{\textit{ Two different notations for conditional probability are used
      throughout the paper, both the conventional notion, P(A|B), and
      subscript notion, $P_B(A)$. At times, the subscript notion can become
      distracting, because it is used to denote conditional as well as
      marginal distributions (pg. 9 for example). For example, on page 16, the
      notation used for the conditional density of $D_k$, defined as
      $p_{Dk}(x | \theta)$, is confusing because it appears that both
      notations are used simultaneously. Adhering to the conventional notion,
      $P(A|B)$, for conditional probabilities would improve readability, and
      eliminate extraneous definitions (on pg. 8 for example). }}

  
  {{ Thank you for this suggestion.  We agree and have
        updated our notation in the revision.  $p_\theta(z_k)$ is now written
        as $p(z_k | \theta)$, and $p_i(z_k)$ as $p(z_k | \Theta \in
        \Lambda_k)$.  We have retained the notation $w_i(\theta)$ for
        compactness and because there is no risk of ambiguity. }}
  

{\textit{When the cost function is introduced on page 8, the
      definition C[i,$\theta$] is puzzling because it is simply defined
      as the cost of choosing hypothesis i, given $\theta$ belonging to
      $\Lambda$ (the entire parameter space). This initially made it
      seem that the cost is uniform across the entire parameter
      space. Later in the paper, section 6 helps clarify the definition
      by showing two different types of cost function. Uniform costs are
      discussed, as well as non-uniform costs, for various cases of
      $\theta$ in $\Lambda_i$. Additional explanation of the cost
      function definition on page 8 would be beneficial. }}

  
    Thank you for this pointing out this need.  We agree and have added
    the following language around the introduction of the cost function:

        \begin{quotation}
          Let $C[i,{\theta}]$ be the cost of choosing $H_i$ when
          ${\theta} \in \Lambda$ is the actual parameter vector.  Note that
          this function is sensitive to a particular value of $\theta$, which
          makes it more general than one that simply assigns a unform cost for
          choosing $H_i$ when $\theta \in \Lambda_j$.  A later section will
          introduce various embodiments of $C[i,{\theta}]$.
        \end{quotation}
  


\section*{Reviewer 3}
\subsection*{Suggestions for Improvement:}

{\textit{Figure 2 can be improved by using some 3D effect on the
      correlation functions.}}

  
   Fig. 2 already includes some features that help the reader view it in
    3 dimensions.  First, it offers a perspective view of a 3D coordinate
    frame in $I$, $Q$, and $\tau$.  Second, in the upper-right inset it allows
    the viewer to appreciate the angular difference between $\xi_{Ak}$ and
    $\xi_{Ik}$ by offering a view along the $\tau$ axis.
  

{\textit{Please explain why the signal models in section II did not
      include a frequency component. With interference or any strong secondary
      signal, it is possible for the tracking loop to diverge which results in
      tracking frequency error. If the error is sufficiently small, it may
      result in a time varying phase.}}

  
   The models in Section II do not include a carrier frequency component,
    since they are expressed in a complex baseband representation.  But they
    do include a Doppler frequency component: $\theta_A$ and $\theta_I$ are
    time varying, as noted in the original text:
    \begin{quotation}
      $P_{\rm A}$, $\tau_{\rm A}$, and $\theta_{\rm A}$ are assumed to vary
      with time, but their time dependency is suppressed for notational
      simplicity.
    \end{quotation}
    Also, the relative time variation between $\theta_{\rm A}$ and
    $\hat{\theta}$ -- the variation that could cause tracking stress -- is
    explicitly noted later on by the notation $\theta_{{\rm A}k}$.
  

{\textit{p6: Technically, R(tau) is a cross correlation function.}}
  
  {{Agreed.  But the technical distiction was already noted in the
        original manuscript:}}
    \begin{quotation}
      The function $R(\tau) = E[C_r(t)C_{\ell}(t - \tau)]$, often called the
      autocorrelation function of $C_r(t)$ even though, strictly speaking,
      $C_{\ell}(t)$ may be slightly different from $C_r(t)$, approximates the
      interaction between $C_r(t)$ and $C_{\ell}(t)$ over the correlation and
      accumulation operations.
    \end{quotation}    
  
  
{\textit{section VIII: How is the performance compared against
      other published methods?}}

  
   Of the various signal authentication techniques mentioned in the
    paper's introduction (which offers an up-to-date summary of techniques),
    this paper's approach can be fairly compared only with a small set of
    practical near-term GNSS signal authentication techniques that do not
    require changes to GNSS signals-in-space, are receiver-autonomous,
    low-cost, require no additional hardware, and can be implemented via a
    software or firmware update. Existing techniques in this category include
    monitoring the total received power via the Automatic Gain Control (AGC)
    setpoint~\cite{akos2012s}, and monitoring autocorrelation profile
    distortion~\cite{ledvina2010cgr, cavaleri2010dscc, huang2016gnss}.  These
    tecniques can be viewed as special cases of the current paper's technique.
    The ``blindspot'' from which these techniques suffer due to their ignoring
    one or the other of received power or correlation function distortion can
    be appreciated emph{a priori}: A received power monitor that ignores
    correlation distortion may not detect a low-power spoofer. Moreover,
    because a power-monitoring-only technique does not distinguish between
    spoofing and jamming, its alarm rate can become intolerable in urban areas
    where so-called personal privacy devices (PPDs, small GNSS
    jammers)~\cite{mitch2012kyejam} are common.  For their part, the
    distortion-monitoring approaches in \cite{ledvina2010cgr,
      cavaleri2010dscc, huang2016gnss} ignore total received power, and thus
    can be fooled by a spoofer transmitting with a significant power advantage
    over the authentic signals, which, by action of the AGC, forces the
    authentic signals under the noise floor, leaving a distortion-free
    correlation function~\cite{humphreys2011ds}.

    This \emph{a priori} comparison is included in the paper's original
    manuscript.  It is also clear from the optimal decision regions shown in
    Fig. 7 that ignoring one or the other of $P_k$ or $D_k$ (as the prior work
    in this area does), would be sub-optimal.  Therefore, we believe that the
    paper convincingly establishes that its approach is superior to other
    approaches in the relevant category for comparison.
  

  
  
{\textit{More details on the test data are expected.}}

  
  {{Thank you. An additional reference for the TEXBAT data set was
        included.  Between the three references to the TEXBAT data, and the
        two references on the ``RNL Multipath and Interference Recordings,''
        we believe the test data are now adequately detailed.}}
  

{\textit{I am not sure that the paper includes enough results to
      show how well the proposed method functions with weak signals. For
      example, if the target signal is weak, interference from other
      GNSS signals may also confuse the detector, since it can be
      observed as multipath or interference. It may trigger more false
      alarms.}}

  
  {{The reviewer raises a good point here: it is possible that near-far
        effects (interaction of weak signals with very strong signals due to
        imperfect orthogonality of their spreading codes) could cause
        distortion that would lead to false spoofing alarms.  The reviewer
        encourages testing to determine if the proposed method works in
        near-far cases with a large gap between the power of the strongest and
        weakest signal.  In fact, the clean data set {\tt cd0}, {\tt wd0}, and
        {\tt wd1} do include a fairly wide range of signal powers.  The most
        powerful signals show $C/N_0 \approx 52$ dB-Hz, while the weakest
        signals are at 35 dB-Hz.  This constitutes 17-dB difference in signal
        power, yet no false spoofing alarms were raised (cf. Table III).  An
        even wider dynamic range could be tested, but the reviewer is
        requested to keep in mind that the difficulty of distinguishing
        multipath from low-power spoofing when severe shadowing is present was
        already treated in Section III:

        \begin{quotation}
          On the other hand, the severe shadowing experienced by mobile
          receivers can cause $\eta > 1$, especially in urban environments
          where surrounding buildings simultaneously attenuate and reflect
          authentic signals \cite{steingass2004measuring}.  In these
          situations, there is no practical upper limit on $\eta$ because the
          direct-path authentic signal may be attenuated by 50 dB or more. It
          is not possible to reliably distinguish multipath from low-power
          spoofing in such circumstances.

          One can avoid this difficulty by applying this paper's detection
          test only when the authentic signal $r_{\rm A}(t)$ is received
          without severe shadowing, in which case the probability that
          $\eta > 1$ under $H_1$ again becomes negligible. In particular, the
          multipath model developed below assumes that $r_{\rm A}(t)$ is
          attenuated less than 6 dB by shadowing.  In practice, one simply
          excludes cases in which the received power $P_k$ is not unusually
          high yet the measured carrier-to-noise ratio $C/N_0$ drops by more
          than 6 dB from its modeled value for an unobstructed authentic
          signal.
        \end{quotation}

        In other words, our approach avoids the effects of severe shadowing by
        applying the PD detector only to signals that have been attenuated
        less than 6 dB by shadowing.  With this exclusion, and with the
        false-alarm-free performance of the paper's methods on the three $H_0$
        data sets, in additions to thorough simulation testing in Section VI,
        we believe that the spoofing false alarm rate under $H_0$ will be
        negligible.}}
  


{\textit{For the power metric introduced in IV.A, it is not
      entirely clear how it can be implemented. How precise does
      the power measurement has to be in order for this method
      to be applicable? In practice, we use carrier to noise
      ratio to gauge signal strength. Depending on the
      implementation choice, it may not be very precise for weak
      signals, or signals with interference. }}

  
   Details on how to measure $P_k$ in practice are included in this
    paragraph from Section IV.A:

    \begin{quotation}
      Receivers with sufficient dynamic range in the discrete samples produced
      from $r_{\rm C}(t)$ do not require an AGC and so can compute received
      power directly by averaging the squared modulus of these samples.  Let
      $W_{P}$ be the bandwidth over which $P_k$ is to be measured.  If $W_P$
      is narrower than the receiver front-end's native bandwidth, then
      $r_{\rm C}(t)$ must be filtered to isolate the desired spectral
      interval.  Fig. 3 shows 2- and 8-MHz bands for $P_k$,
      with Fig. 4 showing the corresponding power time
      histories.  In a receiver whose front end is equipped with an AGC, $P_k$
      is measured indirectly through the AGC setpoint \cite{akos2012s}.  In
      this case $W_{P}$ is equivalent to the front-end noise-equivalent
      bandwith.
    \end{quotation}

    Details on measuring the power indirectly through the AGC setpoint are
    found in \cite{akos2012s}, referenced within this paragraph.

      Note that measuring aggregate received power is different from measuring
      $C/N_0$.  Consider that, if effective $N_0$ rises in a spoofing attack
      commensurate with carrier power $C$, then $C/N_0$ will remain constant
      even as $P_k$ rises.

      How accurately does $P_k$ need to be measured to be useful?  We have
      modified the manuscript in two places to indicate that the paper's
      detection test assumes $\sigma_P = 0.4$ dB.  This is roughly four times
      the standard deviation due to natural variation in a high quality
      receiver in a quiet RF environment (see the top panel in Fig. 4 and the
      discussion within the text).  Receivers whose measurement of $P_k$ is
      significantly less precise than this, due to quantization or inaccuracy
      in the AGC set point-to-$P_k$ mapping, will see their ability to detect
      low-power spoofing and jamming reduced.

      Note that \cite{akos2012s} showed that the precision of the AGC
      measurement as a proxy for in-band power is much better than the natural
      variation in the AGC (see Fig. 7 in \cite{akos2012s}).  This indicates
      that $P_k$ can be reliably obtained from the AGC measurement (to within
      a constant offset that can be calibrated in a quiet RF environment).
  

{\textit{Eq 10 assumes that the code phase can be observed
      by maximizing correlation function. Ideally this
      assumption would be nice to have. However, with weak
      signals or limited bandwidth, it is usually not practical
      to do that. The early-late code discriminator is likely to
      be more sensitive than the peak point in some
      scenarios. What is the sensitivity level of the detector
      against errors made in this step? }}

  
   This paper's assumed recipe for determination of $\hat{\tau}$ is given
    by \[ \hat{\tau} = \arg \max_\tau \xi_k(\tau)  \]  But this recipe is
    only used for modeling purposes.  As the reviewer notes, in practice the
    recipe will be some type of early-late code discriminator, possibly with
    multiple early and late taps.

    The actual receiver used to generate the observables that were processed
    by the PD detector to yield the results in Section VII, the GRID
    science-grade software-defined receiver \cite{lightsey2013demonstration},
    does in fact use an early-minus-late discriminator to obtain $\hat{\tau}$.
    The fact that the PD detector worked well when using the GRID-produced
    observables---even in the case of signals as weak as $35$
    dB-Hz---indicates that the method is not highly sensitive to how
    $\hat{\tau}$ is obtained.
  
\bibliographystyle{IEEEtran} 
\bibliography{pangea}
\end{document}